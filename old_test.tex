
Research in post-quantum cryptography (PQC) has been boosted since the announcement in 
\href{https://csrc.nist.gov/projects/post-quantum-cryptography/post-quantum-cryptography-standardization/call-for-proposals}
{2016 of the National Institute of Standards and Technology (NIST) PQC standardization project}. The development path of 
PQC schemes, from design to implementation, can be minimally divided into four stages. The first 
stage focuses on the hardness of the underlying mathematical problems. The second stage combines 
these mathematical problems into trapdoors to create the actual schemes. The third stage 
concentrates on developing the algorithms and coding prototypes, culminating in the 
release of specifications for the standard algorithm. Finally, the fourth state can be 
summarized as the deployment and widespread knowledge of the cryptographic scheme. 
While many schemes are currently in stage three or four (such as Kyber and Dilithium), 
PQC research is far from being ``done'': 
\textbf{there are still plenty of open problems and challenges in PQC}. 
This is especially true after 2023 when
 \href{https://www.nist.gov/news-events/news/2023/07/nist-announces-additional-digital-signature-candidates-pqc-standardization}
 {NIST announced new candidates for post-quantum signatures}. Additionally, there is a need to improve efficiency and fortify 
 implementations against side-channel attacks (SCA). Finally, it is crucial to deploy PQC and adapt it to 
 current applications such as communication protocols, hardware secure modules (HSM), and several other
  scenarios (IoT, vehicular communication).

In this research project, I will delineate the challenges that post-quantum cryptography (PQC) confronts in terms of its 
security evaluation regarding its physical aspects. While presenting these challenges, I will put forth a 
proposal outlining strategies to address the evaluation of security and measurement of the countermeasures against side-channel 
attacks. 

