
\resizebox{\linewidth}{!}{
\begin{tabular}{|c|c|c|c|}
 \toprule
 Start & End &
  Institution & 
 Position and status\\ 
  \midrule
  01/10/2024  & Current & INRIA & ISFP (Cryptography Researcher) \\
  01/06/2022   & 30/09/2024 & Qualcomm & Senior Cryptographer \\
  01/12/2020     & 30/05/2022   & INRIA Saclay & Post Doc \\
  01/11/2019      & 30/11/2020   & Chalmers University of Technology 
                               & Post Doc \\
 01/11/2015      & 12/11/2019   & Technische Universiteit Eindhoven 
                               & Ph.D. Candidate \\
 01/09/2018     & 01/12/2018   & CryptoExperts 
                               & Internship \\
 01/02/2017     & 01/05/2017   & Riscure 
                               & Internship \\
 01/10/2014     & 31/10/2015   & Bry Tecnologia 
                               & Software Engineer  \\   
\bottomrule
\end{tabular}
}



\begin{table}[h]
    \centering
     \caption{Conference Involvement}
    \label{tab:conference_involvement}
    \renewcommand{\arraystretch}{1.3}
    \setlength{\tabcolsep}{10pt}
    \begin{tabular}{ll}
        \toprule
        \textbf{Role} & \textbf{Conferences and Years} \\
        \midrule
        \multirow{9}{*}{\textbf{Program Committee Member}} & AsiaCCS: 2025 \\
        & Communications in Cryptology: 2025 \\
        & CBCrypto: 2020, 2021 \\
        & CHES: 2022, 2023, 2024 \\
        & Eurocrypt: 2022 \\
        & LatinCrypt: 2023, 2025 \\
        & Asiacrypt: 2023 \\
        & ACNS: 2024 \\
        & PQCrypto: 2025 \\
        \midrule
        \multirow{6}{*}{\textbf{External Reviewer}} & CRYPTO: 2022 \\
        & Asiacrypt: 2018, 2019, 2020, 2021 \\
        & FSE: 2021 \\
        & LatinCrypt: 2021 \\
        & SPACE: 2020 \\
        & PQCrypto: 2018 \\
        \bottomrule
    \end{tabular}
\end{table}

\subsection*{Supervision}
\paragraph{Master Thesis}
~\\
Iggy van Hoof, \textit{Concrete quantum-cryptanalysis of binary elliptic curves}, Eindhoven University of Technology, 2019.

\paragraph{Bachelor Thesis}
~\\
Sigurjon Agustsson, \textit{Montgomery Reduction in RSA}, École Polytechnique, 2021.
~\\
David Brandberg, Lisa Fahlbeck, Henrik Hellstr\"om, Hampus Karlsson, John Kristoffersson, Lukas Sandman, \textit{End-to-end Encrypted Instant Messaging Application}, Chalmers University of Technology, 2020.

\paragraph{Intern at Qualcomm} 
~\\
Liana Koleva, \textit{Vectorization of HQC on RISC-V architecture}, 2023. 


\section*{Selected Publications}

For a full list of publications see: \href{https://scholar.google.com/citations?user=0AIDhMwAAAAJ&hl=fr}{Google Scholar}, 
\href{https://cryptme.in/#publications}{Personal Website} or \href{https://dblp.org/pid/150/9441.html}{DBLP}.

\begin{enumerate}\small
%\item Gustavo Banegas and Ricardo Villanueva-Polanco. On recovering block cipher secret keys in the cold boot attack setting. \textit{Cryptography and Communications}, 2023.
%\item Gustavo Banegas, Valerie Gilchrist, and Benjamin Smith. Efficient supersingularity testing over $\mathbb{GF}(p)$ and CSIDH key validation. \textit{Mathematical Cryptology}, 2(1), 21-35, 2022.
\item Estuardo Alpirez Bock, Gustavo Banegas, Chris Brzuska, Łukasz Chmielewski, Kirthivaasan Puniamurthy, and Milan Šorf. Breaking DPA-protected Kyber via the pair-pointwise multiplication. \textit{ACNS 2024. Lecture Notes in Computer Science}, vol 14584.
\item Gustavo Banegas, Valerie Gilchrist, Anaëlle Le Dévéhat, and Benjamin Smith. Fast and Frobenius: Rational isogeny evaluation over finite fields. \textit{LATINCRYPT 2023. Lecture Notes in Computer Science}, vol 14168.
\item Gustavo Banegas, Daniel J. Bernstein, Fabio Campos, Tung Chou, Tanja Lange, Michael Meyer, Benjamin Smith, and Jana Sotáková. CTIDH: Faster constant-time CSIDH. \textit{IACR Transactions on Cryptographic Hardware and Embedded Systems}, 2021(4):351–387, 2021.
\item Gustavo Banegas, Daniel J. Bernstein, Iggy van Hoof, and Tanja Lange. Concrete quantum cryptanalysis of binary elliptic curves. \textit{IACR Transactions on Cryptographic Hardware and Embedded Systems}, 2021(1):451–472, 2020.
%\item Gustavo Banegas, Paulo S. L. M. Barreto, Edoardo Persichetti, and Paolo Santini. Designing efficient dyadic operations for cryptographic applications. \textit{Journal of Mathematical Cryptology}, 14(1):95–109, 2020.
%\item Bei Liang, Gustavo Banegas, and Aikaterini Mitrokotsa. Statically aggregate verifiable random functions and application to e-lottery. \textit{Cryptography}, 4(4), 2020.
%\item Georgia Tsaloli, Gustavo Banegas, and Aikaterini Mitrokotsa. Practical and provably secure distributed aggregation: Verifiable additive homomorphic secret sharing. \textit{Cryptography}, 4(3):25, 2020.
\item Gustavo Banegas, Paulo S. L. M. Barreto, Brice Odilon Boidje, Pierre-Louis Cayrel, Gilbert Ndollane Dione, Kris Gaj, Cheikh Thiécoumba Gueye, Richard Haeussler, Jean Belo Klamti, Ousmane Ndiaye, Duc Tri Nguyen, Edoardo Persichetti, and Jefferson Ricardini. DAGS: Key encapsulation using dyadic GS codes. \textit{Journal of Mathematical Cryptology}, 12(4):221–239, 2018.
%\item Gustavo Banegas, Ricardo Custódio, and Daniel Panario. A new class of irreducible pentanomials for polynomial-based multipliers in binary fields. \textit{Journal of Cryptographic Engineering}, Online first:1–15, 2018.
%\item Gustavo Banegas and Florian Caullery. Multi-armed SPHINCS+. \textit{ACNS 2023. Lecture Notes in Computer Science}, vol 13907.
%\item Simona Samardjiska, Paolo Santini, Edoardo Persichetti, and Gustavo Banegas. A reaction attack against cryptosystems based on LRPC codes. \textit{LATINCRYPT 2019}, pp. 197–216.
\item Gustavo Banegas and Daniel J. Bernstein. Low-communication parallel quantum multi-target preimage search. \textit{SAC 2017. Lecture Notes in Computer Science}, vol 10719, pp. 325–335.
%\item Gustavo Banegas and Ricardo Villanueva-Polanco. A Fault Analysis on SNOVA. \textit{Cryptology ePrint Archive}, Paper 2024/1883. \url{https://ia.cr/2024/1883}
%\item Gustavo Banegas, Thomas Debris-Alazard, Milena Nedeljković, and Benjamin Smith. Wavelet: Code-based post-quantum signatures with fast verification on microcontrollers. \textit{Cryptology ePrint Archive}, Report 2021/1432. \url{https://ia.cr/2021/1432}
\end{enumerate}
In cryptography, it is common to author list in alphabetical order. We usually follow the cultural statement of \href{https://www.ams.org/profession/leaders/CultureStatement04.pdf}{American Mathematical Society}. 

\subsection*{Software}

\small
\begin{itemize}
\setlength{\itemsep}{1pt}
    \item \textbf{WAVE}: \href{https://github.com/wavesign/wave}{\texttt{github.com/wavesign/wave}}
    \item \textbf{Wavelet}: \href{https://github.com/wavelet/}{\texttt{github.com/wavelet/}}
    \item \textbf{CTIDH}: \href{http://ctidh.isogeny.org/software.html}{\texttt{ctidh.isogeny.org/software.html}}
    \item \textbf{DAGS Key Encapsulation}: \href{https://github.com/gbanegas/dags_v2}{\texttt{github.com/gbanegas/dags\_v2}}
    \item \textbf{HSS/LMS Hash-Based Signatures}: \href{https://github.com/gbanegas/sphss}{\texttt{github.com/gbanegas/sphss}}
    \item \textbf{More Code}: \href{https://github.com/gbanegas/}{\texttt{github.com/gbanegas/}}
\end{itemize}


