
\documentclass{article}
\renewcommand{\thesection}{}
\renewcommand{\thesubsection}{}
\renewcommand{\familydefault}{\sfdefault}
\paperwidth 21cm
\paperheight 29.7cm
\textwidth 16.7cm
\textheight25.7cm
\oddsidemargin0cm \evensidemargin0cm
\topmargin-2cm
\parindent0cm
\parskip0.1cm
\usepackage{graphicx}
\pagestyle{empty}

\begin{document}

\title{Composition du Dossier}
\author{}
\date{}
\maketitle

Le dossier est composé des pièces suivantes :

\begin{enumerate}
    \item \textbf{Le résumé du scientifique} (3 pages max.) mettant en avant les rubriques suivantes, en lien avec les critères d’évaluation :
    \begin{itemize}
        \item \textbf{Présentation} : positionnement, enjeux, objectifs, méthodes, liens avec la stratégie de l’École.
        \item \textbf{Impacts, retombées et ambitions} : publications, colloques, collaborations, contrat industriel, obtention de financement (ERC, ANR, …).
    \end{itemize}
    
    \item \textbf{Le calendrier} détaillant le programme de travail sur 3 ans (1 page max.).
    
    \item \textbf{Le budget prévisionnel} sur 3 ans (1 page max.). Ce budget doit être réaliste, la Fondation se réservant le droit de suspendre, voire d’arrêter le financement du projet notamment en cas de non-respect du budget de manière non motivée.
    
    \item \textbf{Le CV du candidat} (3 pages max.).
\end{enumerate}

\end{document}

