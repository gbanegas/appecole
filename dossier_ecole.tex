
\documentclass[11pt, a4paper]{article}

\usepackage{graphicx}
\usepackage{helvet}
\usepackage[utf8]{inputenc}
\renewcommand\familydefault{\sfdefault}
\usepackage{pgfgantt}       % Pour les diagrammes de Gantt
\usepackage{xcolor}
\usepackage{pdflscape}
\usepackage{url}

\pagestyle{empty}

\newcommand*{\cventry}[7][.25em]{}
\newcommand*{\cvitem}[3][.25em]{}

\ganttset{group/.append style={orange},
milestone/.append style={red},
progress label node anchor/.append style={text=red}}


\begin{document}

\title{APP}
\author{Gustavo Banegas}
\date{}
\maketitle

\section{Résumé du Scientifique}
\begin{enumerate}
    \item \textbf{Le résumé du scientifique} (3 pages max.) mettant en avant les rubriques suivantes, en lien avec les critères d’évaluation :
    \begin{itemize}
        \item \textbf{Présentation} : positionnement, enjeux, objectifs, méthodes, liens avec la stratégie de l’École.
        \item \textbf{Impacts, retombées et ambitions} : publications, colloques, collaborations, contrat industriel, obtention de financement (ERC, ANR, …).
    \end{itemize}
\end{enumerate}

\begin{landscape}
\section{Calendrier}
\begin{ganttchart}[%Specs
     y unit title=0.5cm,
     y unit chart=0.7cm,
     vgrid,hgrid,
     title height=1,
%     title/.style={fill=none},
     title label font=\bfseries\footnotesize,
     bar/.style={fill=blue},
     bar height=0.7,
%   progress label text={},
     group right shift=0,
     group top shift=0.7,
     group height=.3,
     group peaks width={0.2},
     inline]{1}{36}
    %labels
    \gantttitle{A three-years project}{36}\\  % title 1
    \gantttitle[]{2013}{12}                 % title 2
    \gantttitle[]{2015}{12} 
    \gantttitle[]{2016}{12}\\              
    \gantttitle{Q1}{3}                      % title 3
    \gantttitle{Q2}{3}
    \gantttitle{Q3}{3}
    \gantttitle{Q4}{3}
    \gantttitle{Q1}{3}
    \gantttitle{Q2}{3}
    \gantttitle{Q3}{3} 
    \gantttitle{Q4}{3}
    \gantttitle{Q1}{3}
    \gantttitle{Q2}{3}
    \gantttitle{Q3}{3} 
    \gantttitle{Q4}{3}\\
    % Setting group if any
    \ganttgroup[inline=false]{Group 1}{1}{5}\\ 
    \ganttbar[progress=10,inline=false]{Planning}{1}{4}\\
    \ganttmilestone[inline=false]{Milestone 1}{9} \\

    \ganttgroup[inline=false]{Group 2}{6}{12} \\ 
    \ganttbar[progress=2,inline=false]{test1}{10}{19} \\
    \ganttmilestone[inline=false]{Milestone 2}{17} \\
    \ganttbar[progress=5,inline=false]{test2}{11}{20} \\
    \ganttmilestone[inline=false]{Milestone 3}{22} \\       

    \ganttgroup[inline=false]{Group 3}{13}{24} \\ 
    \ganttbar[progress=90,inline=false]{Task A}{13}{15} \\ 
    \ganttbar[progress=50,inline=false, bar progress label node/.append style={below left= 10pt and 7pt}]{Task B}{13}{24} \\ \\
    \ganttbar[progress=30,inline=false]{Task C}{15}{16}\\ 
    \ganttbar[progress=70,inline=false]{Task D}{18}{20} \\ 

    \ganttgroup[inline=false]{Group 3}{25}{36} \\ 
    \ganttbar[progress=90,inline=false]{Task A}{25}{28} \\ 
    \ganttbar[progress=50,inline=false, bar progress label node/.append style={below left= 10pt and 7pt}]{Task B}{27}{28} \\ \\
    \ganttbar[progress=30,inline=false]{Task C}{25}{36}\\ 
\end{ganttchart}
\end{landscape}
\newpage
\section{CV}
\begin{enumerate}
    \item The scientific summary (maximum 3 pages) highlighting the following sections, in connection with the evaluation criteria:
    \begin{itemize}
        \item \textbf{Presentation:} positioning, challenges, objectives, methods, links with the School's strategy.
        \item \textbf{Impacts, outcomes, and ambitions:} publications, conferences, collaborations, industrial contracts, funding acquisition (ERC, ANR, ...).
    \end{itemize}
    
    \item The timeline detailing the work plan over 3 years (maximum 1 page).
    
    \item The projected budget over 3 years (maximum 1 page). This budget must be realistic, and the Foundation reserves the right to suspend or even terminate the project's funding, particularly in the event of an unjustified failure to comply with the budget.
    
    \item The candidate's CV (maximum 3 pages).
\end{enumerate}

\subsection*{Work Experience}

\begin{itemize}
    \item \textbf{Researcher}, \textsc{INRIA}, Palaiseau, France (Oct/2024 -- Current)
    \begin{itemize}
        \item Conduct cryptography research in the following areas, among others:
        \begin{itemize}
            \item Secure implementation of post-quantum cryptography.
            \item Design and development of specialized hardware for post-quantum cryptography.
            \item Creation of countermeasures to mitigate side-channel vulnerabilities.
        \end{itemize}
    \end{itemize}

    \item \textbf{Senior Cryptographer}, \textsc{Qualcomm}, Sophia Antipolis, France (Jul/2022 -- Sept/2024)
    \begin{itemize}
        \item Development of post-quantum cryptography on Snapdragon processors, including but not limited to:
        \begin{itemize}
            \item Design and develop specific hardware for post-quantum cryptography.
            \item Development of new attacks on post-quantum cryptography (side-channel attacks).
            \item Development of countermeasures against side-channel attacks.
            \item Speed-up implementations on Cortex-M3 and M4.
            \item Development of post-quantum cryptography for RISC-V. 
        \end{itemize}
    \end{itemize}

    \item \textbf{Post-doc}, \textsc{INRIA and \'{E}cole polytechnique}, Paris, France (Dec/2020 -- Jul/2022)
    \begin{itemize}
        \item Development of post-quantum cryptography in embedded devices:
        \begin{itemize}
            \item Development of new attacks on post-quantum cryptography (side-channel attacks).
            \item Development of countermeasures against side-channel attacks.
            \item Speed-up implementations of cryptographic signatures for RIOT-OS.
        \end{itemize}
    \end{itemize}

    \item \textbf{Post-doc}, \textsc{Chalmers University of Technology}, Gothenburg, Sweden (Nov/2019 -- Nov/2020)
    \begin{itemize}
        \item Development of the WASP Project:
        \begin{itemize}
            \item Development of new attacks on post-quantum cryptography.
            \item Development of post-quantum cryptography.
            \item Development of verifiable functions.
        \end{itemize}
    \end{itemize}

    \item \textbf{Research Assistant}, \textsc{Chalmers University of Technology}, Gothenburg, Sweden (Sep/2019 -- Nov/2019)
    \begin{itemize}
        \item Development of the WASP Project:
        \begin{itemize}
            \item Development of new attacks on post-quantum cryptography.
            \item Development of post-quantum cryptography.
            \item Development of verifiable functions.
        \end{itemize}
    \end{itemize}

    \item \textbf{Intern}, \textsc{Cryptoexperts}, Paris, France (Sep/2018 -- Nov/2018)
    \begin{itemize}
        \item Side-channel attacks on post-quantum cryptography implementations.
        \begin{itemize}
            \item Detected leakage of timing in operations to develop timing attacks.
        \end{itemize}
    \end{itemize}

    \item \textbf{Intern}, \textsc{Riscure}, Delft, Netherlands (Feb/2017 -- Apr/2017)
    \begin{itemize}
        \item Side-channel attacks on ECC implementations.
        \begin{itemize}
            \item Investigated attacks on implementations of ECC in FPGAs using power analysis.
        \end{itemize}
    \end{itemize}

    \item \textbf{System Analyst}, \textsc{BRy Tecnologia}, Florian\'{o}polis, Brazil (Oct/2014 -- Sep/2015)
    \begin{itemize}
        \item Software development for Public Key Infrastructure (PKI).
        \begin{itemize}
            \item Developed software in Java and C++.
            \item Integrated HSM in Java applications.
            \item Managed a team using Scrum.
        \end{itemize}
    \end{itemize}

    \item \textbf{Researcher, Project Manager, and Developer}, \textsc{LabSEC - Laboratory for Computer Security}, Florian\'{o}polis, Brazil (Nov/2009 -- Oct/2014)
    \begin{itemize}
        \item Researcher in cryptography, project manager, and developer of security software using \textit{Java}, \textit{C/C++}, and \textit{Python}.
        \begin{itemize}
            \item Researched cryptography applied to PKI.
            \item Managed the project reference for the Brazilian PKI.
            \item Managed the project defining attribute certification in Brazil.
            \item Developed software in C/C++, Java, and Python.
        \end{itemize}
    \end{itemize}

\end{itemize}

\subsection*{Program Committee Member}
\begin{itemize}
    \item CBCrypto: 2020, 2021
    \item CHES: 2022, 2023, 2024
    \item Eurocrypt: 2022
    \item LatinCrypt: 2023
    \item Asiacrypt: 2023
    \item ACNS: 2024
\end{itemize}

\subsection*{External Reviewer}
\begin{itemize}
    \item CRYPTO: 2022
    \item Asiacrypt: 2018, 2019, 2020, 2021
    \item FSE: 2021
    \item LatinCrypt: 2021
    \item SPACE: 2020
    \item PQCrypto: 2018
\end{itemize}

\subsection*{Software}
\begin{itemize}
    \item \textbf{WAVE}: \url{https://github.com/wavesign/wave}
    \item \textbf{Wavelet}: \url{https://github.com/wavelet/}
    \item \textbf{CTIDH}: \url{http://ctidh.isogeny.org/software.html}
    \item \textbf{DAGS Key encapsulation}: \url{https://github.com/gbanegas/dags_v2}
    \item \textbf{HSS/LMS hash-based signatures}: \url{https://github.com/gbanegas/sphss}
    \item \textbf{More code}: \url{https://github.com/gbanegas/}
\end{itemize}

\subsection*{Supervision: Master Theses}
\begin{itemize}
    \item \textbf{Maya--Iggy van Hoof}:  
    \textit{Concrete quantum-cryptanalysis of binary elliptic curves}, Eindhoven University of Technology, 2019.
\end{itemize}

\subsection*{Supervision: Bachelor Theses}
\begin{itemize}
    \item \textbf{Sigurjon Agustsson}:  
    \textit{Montgomery Reduction in RSA}, École Polytechnique, 2021.
    
    \item \textbf{David Brandberg, Lisa Fahlbeck, Henrik Hellström, Hampus Karlsson, John Kristoffersson, Lukas Sandman}:  
    \textit{End-to-end Encrypted Instant Messaging Application}, Chalmers University of Technology, 2020.
\end{itemize}

\nocite{*}
\bibliographystyle{plainyr-rev} 
\bibliography{publications}   

\end{document}

