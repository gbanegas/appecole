
\documentclass[11pt, a4paper]{article}
\usepackage{parselines} 
\usepackage{graphicx}
\usepackage{helvet}
\usepackage[utf8]{inputenc}
\renewcommand\familydefault{\sfdefault}
\usepackage{pgfgantt}       % Pour les diagrammes de Gantt
\usepackage{pdflscape}
\usepackage{url}
\usepackage{makecell}
\usepackage{wrapfig}
\usepackage{xcolor}
\usepackage{listings}
\usepackage{amsfonts}
\usepackage{multicol}
\usepackage{mathtools}
\usepackage{amsmath}
\usepackage{float}
\newcommand{\bigzero}{\mbox{\normalfont\Large\bfseries 0}}
\newcommand{\rvline}{\hspace*{-\arraycolsep}\vline\hspace*{-\arraycolsep}}

\usepackage[linesnumbered,ruled,vlined]{algorithm2e}
\usepackage[colorlinks=true,hyperfootnotes=false,citecolor=blue]{hyperref}
\usepackage[capitalise]{cleveref}
\usepackage{siunitx}
\usepackage{todonotes}
\usepackage{cite}
\usepackage{comment}
\usepackage{booktabs}
\usepackage{multirow}
\usepackage{tabularray}
\UseTblrLibrary{booktabs}
\usepackage{caption}
\usepackage{setspace}

\usepackage{geometry}
\usepackage{tabularx}
\usepackage{enumitem}
\usepackage[table, dvipsnames]{xcolor}
\usepackage[utf8]{inputenc}
\usepackage[T1]{fontenc}
\usepackage{lmodern}
\usepackage{amsmath}

\pagestyle{empty}

\geometry{a4paper, margin=1in}

\ganttset{group/.append style={orange},
milestone/.append style={red},
progress label node anchor/.append style={text=red}}

\usepackage{multicol}
\usepackage{titlesec}

\titleformat{\section}{\large\bfseries}{\thesection}{1em}{}
\titlespacing*{\section}{0pt}{1ex}{1ex}

\usepackage{tikz}
\usetikzlibrary{shapes.geometric, arrows}
\tikzstyle{startstop} = [rectangle, rounded corners, minimum width=3cm, minimum height=1cm,text centered, draw=black, fill=red!30]
\tikzstyle{process} = [rectangle, minimum width=3cm, minimum height=1cm, text centered, draw=black, fill=blue!20]
\tikzstyle{arrow} = [thick,->,>=stealth]

\usepackage{enumitem}
\setlist[enumerate]{itemsep=0pt, parsep=0pt, topsep=0pt}
\setlist[itemize]{itemsep=0pt, parsep=0pt, topsep=0pt}



\begin{document}

\title{APP}
\author{Gustavo Banegas}
\date{}
\maketitle

\section{Résumé du Scientifique}
\begin{enumerate}
    \item \textbf{Le résumé du scientifique} (3 pages max.) mettant en avant les rubriques suivantes, en lien avec les critères d’évaluation :
    \begin{itemize}
        \item \textbf{Présentation} : positionnement, enjeux, objectifs, méthodes, liens avec la stratégie de l’École.
        \item \textbf{Impacts, retombées et ambitions} : publications, colloques, collaborations, contrat industriel, obtention de financement (ERC, ANR, …).
    \end{itemize}
\end{enumerate}

\begin{landscape}
\section{Calendrier}
\begin{ganttchart}[%Specs
     y unit title=0.5cm,
     y unit chart=0.7cm,
     vgrid,hgrid,
     title height=1,
%     title/.style={fill=none},
     title label font=\bfseries\footnotesize,
     bar/.style={fill=blue},
     bar height=0.7,
%   progress label text={},
     group right shift=0,
     group top shift=0.7,
     group height=.3,
     group peaks width={0.2},
     inline]{1}{36}
    %labels
    %\gantttitle{A three-years project}{36}\\  % title 1
    \gantttitle[]{2025}{6}                 % title 2
    \gantttitle[]{2026}{12} 
    \gantttitle[]{2027}{12}
    \gantttitle[]{2028}{6}\\              
 %   \gantttitle{Q1}{3}                      % title 3
 %   \gantttitle{Q2}{3}
    \gantttitle{Q3}{3}
    \gantttitle{Q4}{3}
    \gantttitle{Q1}{3}
    \gantttitle{Q2}{3}
    \gantttitle{Q3}{3} 
    \gantttitle{Q4}{3}
    \gantttitle{Q1}{3}
    \gantttitle{Q2}{3}
    \gantttitle{Q3}{3} 
    \gantttitle{Q4}{3}
    \gantttitle{Q1}{3}                      % title 3
    \gantttitle{Q2}{3}    \\
    % Setting group if any
   % \ganttgroup[inline=false]{Group 1}{1}{5}\\ 
    \ganttbar[inline=false]{Selection of candidates}{1}{4}\\
    \ganttbar[inline=false]{Acquiring equipment}{3}{5}\\
    \ganttbar[inline=false]{Evaluation of candidates}{5}{10}\\
    \ganttbar[inline=false]{Development of attacks}{6}{12}\\
    \ganttbar[inline=false, bar/.append style={fill=teal}]{Intership Master M2}{5}{11}\\
    %\ganttbar[inline=false]{Writing of the attack}{10}{11}\\
    \ganttmilestone[inline=false]{1st Publication (Journal)}{13} \\
    \ganttbar[inline=false, bar/.append style={fill=purple}]{Writing application for ERC}{9}{12} \\
    \ganttmilestone[inline=false, milestone/.append style={fill=green}]{Deadline Submittion ERC 2026}{12} \\
    \ganttbar[inline=false]{Development of the countermeasure}{11}{16}\\
    \ganttbar[inline=false]{Implementation of the countermeasure}{13}{19}\\
    \ganttbar[inline=false, bar/.append style={fill=teal}]{Intership Master M2}{13}{19}\\
    \ganttbar[inline=false]{Evaluation of the countermeasure}{16}{19}\\
    \ganttbar[inline=false]{Scentific write of the framework}{15}{22}\\
    \ganttmilestone[inline=false]{2nd publication (CHES)}{20} \\
    \ganttbar[inline=false]{Evaluation of hardware design }{20}{26}\\
    \ganttbar[inline=false]{Prototype of secure hardware}{22}{32}\\
    \ganttbar[inline=false]{Evaluation of secure techniques}{26}{34}\\
    \ganttmilestone[inline=false]{3nd publication (CHES/Eurocrypt)}{35} \\
    %\ganttbar[inline=false]{Preparatiof for the defense}{33}{36}\\
    %\ganttmilestone[inline=false]{PhD Thesis Defense}{36} \\ 
\end{ganttchart}
\end{landscape}


\newpage
\section{CV}
\begin{enumerate}
    \item The scientific summary (maximum 3 pages) highlighting the following sections, in connection with the evaluation criteria:
    \begin{itemize}
        \item \textbf{Presentation:} positioning, challenges, objectives, methods, links with the School's strategy.
        \item \textbf{Impacts, outcomes, and ambitions:} publications, conferences, collaborations, industrial contracts, funding acquisition (ERC, ANR, ...).
    \end{itemize}
    
    \item The timeline detailing the work plan over 3 years (maximum 1 page).
    
    \item The projected budget over 3 years (maximum 1 page). This budget must be realistic, and the Foundation reserves the right to suspend or even terminate the project's funding, particularly in the event of an unjustified failure to comply with the budget.
    
    \item The candidate's CV (maximum 3 pages).
\end{enumerate}

\subsection*{Work Experience}
\medskip


\subsection*{Professional Experience}
\resizebox{\linewidth}{!}{
\begin{tabular}{|c|c|c|c|}
 \toprule
 Start & End &
  Institution & 
 Position and status\\ 
  \midrule
  01/10/2024  & Current & INRIA & ISFP (Cryptography Researcher) \\
  01/06/2022   & 30/09/2024 & Qualcomm & Senior Cryptographer \\
  01/12/2020     & 30/05/2022   & INRIA Saclay & Post Doc \\
  01/11/2019      & 30/11/2020   & Chalmers University of Technology 
                               & Post Doc \\
 01/11/2015      & 12/11/2019   & Technische Universiteit Eindhoven 
                               & Ph.D. Candidate \\
 01/09/2018     & 01/12/2018   & CryptoExperts 
                               & Internship \\
 01/02/2017     & 01/05/2017   & Riscure 
                               & Internship \\
 01/10/2014     & 31/10/2015   & Bry Tecnologia 
                               & Software Engineer  \\   
\bottomrule
\end{tabular}
}



\subsection*{Scientific Responsabilities}
\begin{table}[h]
    \centering
     \caption{Conference Involvement}
    \label{tab:conference_involvement}
    \renewcommand{\arraystretch}{1.3}
    \setlength{\tabcolsep}{10pt}
    \begin{tabular}{l|l}
        \toprule
        \textbf{Role} & \textbf{Conferences and Years} \\
        \midrule
        \multirow{9}{*}{\textbf{Program Committee Member}} & AsiaCCS: 2025 \\
        & Communications in Cryptology: 2025 \\
        & CBCrypto: 2020, 2021 \\
        & CHES: 2022, 2023, 2024 \\
        & Eurocrypt: 2022 \\
        & LatinCrypt: 2023, 2025 \\
        & Asiacrypt: 2023 \\
        & ACNS: 2024 \\
        & PQCrypto: 2025 \\
        \midrule
        \multirow{6}{*}{\textbf{External Reviewer}} & CRYPTO: 2022 \\
        & Asiacrypt: 2018, 2019, 2020, 2021 \\
        & FSE: 2021 \\
        & LatinCrypt: 2021 \\
        & SPACE: 2020 \\
        & PQCrypto: 2018 \\
        \bottomrule
    \end{tabular}
\end{table}

\subsection*{Supervision}
\paragraph{Master Thesis}
~\\
Iggy van Hoof, \textit{Concrete quantum-cryptanalysis of binary elliptic curves}, Eindhoven University of Technology, 2019.

\paragraph{Bachelor Thesis}
~\\
Sigurjon Agustsson, \textit{Montgomery Reduction in RSA}, École Polytechnique, 2021.
~\\
David Brandberg, Lisa Fahlbeck, Henrik Hellstr\"om, Hampus Karlsson, John Kristoffersson, Lukas Sandman, \textit{End-to-end Encrypted Instant Messaging Application}, Chalmers University of Technology, 2020.

\paragraph{Intern at Qualcomm} 
~\\
Liana Koleva, \textit{Vectorization of HQC on RISC-V architecture}, 2023. 


\section*{Selected Publications}

For a full list of publications see: \href{https://scholar.google.com/citations?user=0AIDhMwAAAAJ&hl=fr}{Google Scholar}, 
\href{https://cryptme.in/#publications}{Personal Website} or \href{https://dblp.org/pid/150/9441.html}{DBLP}.

\begin{enumerate}\small
%\item Gustavo Banegas and Ricardo Villanueva-Polanco. On recovering block cipher secret keys in the cold boot attack setting. \textit{Cryptography and Communications}, 2023.
%\item Gustavo Banegas, Valerie Gilchrist, and Benjamin Smith. Efficient supersingularity testing over $\mathbb{GF}(p)$ and CSIDH key validation. \textit{Mathematical Cryptology}, 2(1), 21-35, 2022.
\item Estuardo Alpirez Bock, Gustavo Banegas, Chris Brzuska, Łukasz Chmielewski, Kirthivaasan Puniamurthy, and Milan Šorf. Breaking DPA-protected Kyber via the pair-pointwise multiplication. \textit{ACNS 2024. Lecture Notes in Computer Science}, vol 14584.
\item Gustavo Banegas, Valerie Gilchrist, Anaëlle Le Dévéhat, and Benjamin Smith. Fast and Frobenius: Rational isogeny evaluation over finite fields. \textit{LATINCRYPT 2023. Lecture Notes in Computer Science}, vol 14168.
\item Gustavo Banegas, Daniel J. Bernstein, Fabio Campos, Tung Chou, Tanja Lange, Michael Meyer, Benjamin Smith, and Jana Sotáková. CTIDH: Faster constant-time CSIDH. \textit{IACR Transactions on Cryptographic Hardware and Embedded Systems}, 2021(4):351–387, 2021.
\item Gustavo Banegas, Daniel J. Bernstein, Iggy van Hoof, and Tanja Lange. Concrete quantum cryptanalysis of binary elliptic curves. \textit{IACR Transactions on Cryptographic Hardware and Embedded Systems}, 2021(1):451–472, 2020.
%\item Gustavo Banegas, Paulo S. L. M. Barreto, Edoardo Persichetti, and Paolo Santini. Designing efficient dyadic operations for cryptographic applications. \textit{Journal of Mathematical Cryptology}, 14(1):95–109, 2020.
%\item Bei Liang, Gustavo Banegas, and Aikaterini Mitrokotsa. Statically aggregate verifiable random functions and application to e-lottery. \textit{Cryptography}, 4(4), 2020.
%\item Georgia Tsaloli, Gustavo Banegas, and Aikaterini Mitrokotsa. Practical and provably secure distributed aggregation: Verifiable additive homomorphic secret sharing. \textit{Cryptography}, 4(3):25, 2020.
\item Gustavo Banegas, Paulo S. L. M. Barreto, Brice Odilon Boidje, Pierre-Louis Cayrel, Gilbert Ndollane Dione, Kris Gaj, Cheikh Thiécoumba Gueye, Richard Haeussler, Jean Belo Klamti, Ousmane Ndiaye, Duc Tri Nguyen, Edoardo Persichetti, and Jefferson Ricardini. DAGS: Key encapsulation using dyadic GS codes. \textit{Journal of Mathematical Cryptology}, 12(4):221–239, 2018.
%\item Gustavo Banegas, Ricardo Custódio, and Daniel Panario. A new class of irreducible pentanomials for polynomial-based multipliers in binary fields. \textit{Journal of Cryptographic Engineering}, Online first:1–15, 2018.
%\item Gustavo Banegas and Florian Caullery. Multi-armed SPHINCS+. \textit{ACNS 2023. Lecture Notes in Computer Science}, vol 13907.
%\item Simona Samardjiska, Paolo Santini, Edoardo Persichetti, and Gustavo Banegas. A reaction attack against cryptosystems based on LRPC codes. \textit{LATINCRYPT 2019}, pp. 197–216.
\item Gustavo Banegas and Daniel J. Bernstein. Low-communication parallel quantum multi-target preimage search. \textit{SAC 2017. Lecture Notes in Computer Science}, vol 10719, pp. 325–335.
%\item Gustavo Banegas and Ricardo Villanueva-Polanco. A Fault Analysis on SNOVA. \textit{Cryptology ePrint Archive}, Paper 2024/1883. \url{https://ia.cr/2024/1883}
%\item Gustavo Banegas, Thomas Debris-Alazard, Milena Nedeljković, and Benjamin Smith. Wavelet: Code-based post-quantum signatures with fast verification on microcontrollers. \textit{Cryptology ePrint Archive}, Report 2021/1432. \url{https://ia.cr/2021/1432}
\end{enumerate}
In cryptography, it is common to author list in alphabetical order. We usually follow the cultural statement of \href{https://www.ams.org/profession/leaders/CultureStatement04.pdf}{American Mathematical Society}. 

\subsection*{Teaching}
\textbf{Special Class} (2021) — Universidade Federal de Santa Catarina (Online), Florianópolis, Brazil  
Taught Quantum Computation, Grover's Algorithm, and Shor's Algorithm. ~\\

\textbf{Special Classes} (2020) — Chalmers University of Technology, Gothenburg, Sweden  
Taught various cryptography topics, replacing Prof. Katerina Mitrokotsa:  
\begin{itemize}
\item RSA and Primality Testing  
\item Attacks on Block Ciphers and Intro to PKC  
\item Block Ciphers and Operation Modes  
\item Sigma Protocols  
\end{itemize}

\textbf{Tutor} (2016--2019) — Technische Universiteit Eindhoven, Netherlands  
Tutor for courses including:  
\begin{itemize}
\item Introduction to Cryptology  
\item Basic Mathematics  
\item Algebra and Discrete Mathematics  
\end{itemize}

\subsubsection*{Grants}
\textbf{Marie Skłodowska-Curie ITN} — \href{https://www.ecrypt.eu.org/net/}{ECRYPT-NET Project}  
Fellow PhD (2015–2019).  
~\\ \\
\textbf{Wallenberg WASP Expedition Project} — Massive, Secure, and Low-Latency Connectivity for IoT Applications  
Fellow Researcher (2019–2020).  


\subsection*{Software}

\small
\begin{itemize}
\setlength{\itemsep}{1pt}
    \item \textbf{WAVE}: \href{https://github.com/wavesign/wave}{\texttt{github.com/wavesign/wave}}
    \item \textbf{Wavelet}: \href{https://github.com/wavelet/}{\texttt{github.com/wavelet/}}
    \item \textbf{CTIDH}: \href{http://ctidh.isogeny.org/software.html}{\texttt{ctidh.isogeny.org/software.html}}
    \item \textbf{DAGS Key Encapsulation}: \href{https://github.com/gbanegas/dags_v2}{\texttt{github.com/gbanegas/dags\_v2}}
    \item \textbf{HSS/LMS Hash-Based Signatures}: \href{https://github.com/gbanegas/sphss}{\texttt{github.com/gbanegas/sphss}}
    \item \textbf{More Code}: \href{https://github.com/gbanegas/}{\texttt{github.com/gbanegas/}}
\end{itemize}




%\nocite{*}
%\bibliographystyle{plainyr-rev} 
%\bibliography{publications}   

\end{document}

