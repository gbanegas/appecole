
\documentclass[11pt, a4paper]{article}

\usepackage{graphicx}
\usepackage{helvet}
\usepackage[utf8]{inputenc}
\renewcommand\familydefault{\sfdefault}
\usepackage{pgfgantt}       % Pour les diagrammes de Gantt
\usepackage{xcolor}
\usepackage{pdflscape}
\pagestyle{empty}

\ganttset{group/.append style={orange},
milestone/.append style={red},
progress label node anchor/.append style={text=red}}


\begin{document}

\title{APP}
\author{Gustavo Banegas}
\date{}
\maketitle

\section{Résumé du Scientifique}
\begin{enumerate}
    \item \textbf{Le résumé du scientifique} (3 pages max.) mettant en avant les rubriques suivantes, en lien avec les critères d’évaluation :
    \begin{itemize}
        \item \textbf{Présentation} : positionnement, enjeux, objectifs, méthodes, liens avec la stratégie de l’École.
        \item \textbf{Impacts, retombées et ambitions} : publications, colloques, collaborations, contrat industriel, obtention de financement (ERC, ANR, …).
    \end{itemize}
\end{enumerate}

\begin{landscape}
\section{Calendrier}
\begin{ganttchart}[%Specs
     y unit title=0.5cm,
     y unit chart=0.7cm,
     vgrid,hgrid,
     title height=1,
%     title/.style={fill=none},
     title label font=\bfseries\footnotesize,
     bar/.style={fill=blue},
     bar height=0.7,
%   progress label text={},
     group right shift=0,
     group top shift=0.7,
     group height=.3,
     group peaks width={0.2},
     inline]{1}{36}
    %labels
    \gantttitle{A three-years project}{36}\\  % title 1
    \gantttitle[]{2013}{12}                 % title 2
    \gantttitle[]{2015}{12} 
    \gantttitle[]{2016}{12}\\              
    \gantttitle{Q1}{3}                      % title 3
    \gantttitle{Q2}{3}
    \gantttitle{Q3}{3}
    \gantttitle{Q4}{3}
    \gantttitle{Q1}{3}
    \gantttitle{Q2}{3}
    \gantttitle{Q3}{3} 
    \gantttitle{Q4}{3}
    \gantttitle{Q1}{3}
    \gantttitle{Q2}{3}
    \gantttitle{Q3}{3} 
    \gantttitle{Q4}{3}\\
    % Setting group if any
    \ganttgroup[inline=false]{Group 1}{1}{5}\\ 
    \ganttbar[progress=10,inline=false]{Planning}{1}{4}\\
    \ganttmilestone[inline=false]{Milestone 1}{9} \\

    \ganttgroup[inline=false]{Group 2}{6}{12} \\ 
    \ganttbar[progress=2,inline=false]{test1}{10}{19} \\
    \ganttmilestone[inline=false]{Milestone 2}{17} \\
    \ganttbar[progress=5,inline=false]{test2}{11}{20} \\
    \ganttmilestone[inline=false]{Milestone 3}{22} \\       

    \ganttgroup[inline=false]{Group 3}{13}{24} \\ 
    \ganttbar[progress=90,inline=false]{Task A}{13}{15} \\ 
    \ganttbar[progress=50,inline=false, bar progress label node/.append style={below left= 10pt and 7pt}]{Task B}{13}{24} \\ \\
    \ganttbar[progress=30,inline=false]{Task C}{15}{16}\\ 
    \ganttbar[progress=70,inline=false]{Task D}{18}{20} \\ 

    \ganttgroup[inline=false]{Group 3}{25}{36} \\ 
    \ganttbar[progress=90,inline=false]{Task A}{25}{28} \\ 
    \ganttbar[progress=50,inline=false, bar progress label node/.append style={below left= 10pt and 7pt}]{Task B}{27}{28} \\ \\
    \ganttbar[progress=30,inline=false]{Task C}{25}{36}\\ 
\end{ganttchart}
\end{landscape}
\newpage
\section{CV}

\begin{enumerate}
    \item \textbf{Le résumé du scientifique} (3 pages max.) mettant en avant les rubriques suivantes, en lien avec les critères d’évaluation :
    \begin{itemize}
        \item \textbf{Présentation} : positionnement, enjeux, objectifs, méthodes, liens avec la stratégie de l’École.
        \item \textbf{Impacts, retombées et ambitions} : publications, colloques, collaborations, contrat industriel, obtention de financement (ERC, ANR, …).
    \end{itemize}
    
    \item \textbf{Le calendrier} détaillant le programme de travail sur 3 ans (1 page max.).
    
    \item \textbf{Le budget prévisionnel} sur 3 ans (1 page max.). Ce budget doit être réaliste, la Fondation se réservant le droit de suspendre, voire d’arrêter le financement du projet notamment en cas de non-respect du budget de manière non motivée.
    
    \item \textbf{Le CV du candidat} (3 pages max.).
\end{enumerate}


\end{document}

